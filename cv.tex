%% start of file `template.tex'.
%% Copyright 2006-2010 Xavier Danaux (xdanaux@gmail.com).
%
% This work may be distributed and/or modified under the
% conditions of the LaTeX Project Public License version 1.3c,
% available at http://www.latex-project.org/lppl/.


\documentclass[11pt,a4paper]{moderncv}

% moderncv themes
%\moderncvtheme[blue]{casual}                 % optional argument are 'blue' (default), 'orange', 'red', 'green', 'grey' and 'roman' (for roman fonts, instead of sans serif fonts)
\moderncvtheme[grey]{classic}                % idem

% character encoding
\usepackage[utf8]{inputenc}                   % replace by the encoding you are using

% adjust the page margins
\usepackage[scale=0.8]{geometry}
%\setlength{\hintscolumnwidth}{3cm}                     % if you want to change the width of the column with the dates
%\AtBeginDocument{\setlength{\maketitlenamewidth}{6cm}}  % only for the classic theme, if you want to change the width of your name placeholder (to leave more space for your address details
%\AtBeginDocument{\recomputelengths}                     % required when changes are made to page layout lengths

% personal data
\firstname{Joao}
\familyname{Azevedo}
\title{Software Engineer}               % optional, remove the line if not wanted
\address{Porto}{Portugal}    % optional, remove the line if not wanted
\mobile{(+351) 91 084 25 79}                    % optional, remove the line if not wanted
%\phone{phone (optional)}                      % optional, remove the line if not wanted
%\fax{fax (optional)}                          % optional, remove the line if not wanted
\email{joao.c.azevedo@gmail.com}                      % optional, remove the line if not wanted
\homepage{https://jcazevedo.net/}                % optional, remove the line if not wanted
%\extrainfo{additional information (optional)} % optional, remove the line if not wanted
\photo[64pt]{picture}                         % '64pt' is the height the picture must be resized to and 'picture' is the name of the picture file; optional, remove the line if not wanted
%\quote{Some quote (optional)}                 % optional, remove the line if not wanted

% to show numerical labels in the bibliography; only useful if you make citations in your resume
\makeatletter
\renewcommand*{\bibliographyitemlabel}{\@biblabel{\arabic{enumiv}}}
\makeatother

% bibliography with mutiple entries
%\usepackage{multibib}
%\newcites{book,misc}{{Books},{Others}}

%\nopagenumbers{}                             % uncomment to suppress automatic page numbering for CVs longer than one page
%----------------------------------------------------------------------------------
%            content
%----------------------------------------------------------------------------------
\begin{document}
\maketitle

\section{Experience}
\cventry{2012--}{Software Engineer}{ShiftForward}{Porto, Portugal}{}{}%

\cventry{2011--2012}{Researcher}{Fraunhofer AICOS}{Porto, Portugal}{}{Designed and developed the new version of Mover, an application to promote mobility among elder people. Implemented the mobile frontend in Android and the backend in Ruby on Rails. Developed several signal analysis tools in Java for \href{https://play.google.com/store/apps/details?id=pt.fraunhofer.dancedontfall}{Dance! Don't Fall}, a dancing game for elders that was ranked in the top 25 apps in the \href{http://mobileappsshowdown.com/}{Mobile Apps Showdown} of \href{http://www.cesweb.org/}{CES} 2012. Researched various state-of-the-art technologies, protocols, norms and standards to be used within the \href{http://www.aal4all.org/}{AAL4ALL} project, a project aiming to develop an ecossystem of products and services for Ambient Assisted Living associated to a business model and validated through a large scale trial. Researched machine learning techniques for time series analysis, focused on the problem of fall detection and activity classification.\newline{}}%

\cventry{2010--2011}{Software Engineer}{SISCOG}{Lisbon, Portugal}{}{Implemented several new features and performed corrective maintenance in a crew-scheduling application for VR, a Finnish railway company. Wrote several specifications for new features to be implemented. Performed various code reviews. Developed solutions to enable the integration of git as a version control system in the company's workflow and to automate some of the company's internal processes. Worked mainly with Common Lisp and the Common Lisp Object System.\newline{}}

\cventry{2009--2010}{Student Researcher}{Faculdade de Engenharia da Universidade do Porto}{Porto, Portugal}{}{Designed a module based architecture for a configurable Inductive Logic Programming system development environment. Developed a small working prototype using Java and Prolog. Wrote a module for the \href{http://www.dcc.fc.up.pt/~vsc/Yap/}{YAP} high-performance Prolog compiler to enable the access to \href{http://www.r-project.org/}{R-project} facilities from within the Prolog engine.\newline{}}

\cventry{2007--2010}{Software Developer}{IMAGE}{Porto, Portugal}{}{Made part of the development team of the browser-based game IMAGE - Industrial Management Game, started on the PESC (Projectar, Empreender e Saber Concretizar) projects initiative at Faculdade de Engenharia da Universidade do Porto. Worked mainly on the Ruby on Rails backend that supported the game, both by introducing new features and refactoring a legacy codebase. Made part of the selection process for new students to integrate the project.\newline{}}

\cventry{2010}{Undergraduate Teaching Assistant}{Faculdade de Engenharia da Universidade do Porto}{Porto, Portugal}{}{Supported first-year students in a Programming course. Provided assistance on the doubts the students had regarding the coursework on practical classes. Taught basic object-oriented programming with C++. Wrote the assignments for the semester.\newline{}}

\cventry{2009}{Junior Engineer}{Critical Software, S.A.}{Coimbra, Portugal}{}{Developed a solution to support software reuse in the company's projects, unified with the current infrastructure regarding the version control systems and the intranet's information system. Worked mainly with Java, using the Stripes framework, and MSSQL.\newline{}}

\cventry{2009}{Undergraduate Teaching Assistant}{Faculdade de Engenharia da Universidade do Porto}{Porto, Portugal}{}{Supported first-year students in a Programming course. Provided assistance on the doubts the students had regarding the coursework on practical classes. Taught basic object-oriented programming with C++. Wrote the assignments for the semester.\newline{}}

\cventry{2008}{Undergraduate Teaching Assistant}{Faculdade de Engenharia da Universidade do Porto}{Porto, Portugal}{}{Supported second-year students in a Computer Graphics course. Provided assistance on the doubts the students had regarding the coursework on practical classes. The main technologies involved were OpenGL with C++ and Java Swing.\newline{}}

\section{Education}
\cventry{2005--2010}{Integrated Masters in Informatics and Computing Engineering}{Faculdade de Engenharia da Universidade do Porto}{Porto, Portugal}{}{Grade average of 18 out of 20\newline{}}
\cventry{2002--2005}{Secondary Education -- Natural--Scientific Option}{Escola Secund\'{a}ria de Carvalhos}{Vila Nova de Gaia, Portugal}{}{Grade average of 18 out of 20.\newline{}}

\section{Independent Coursework}
\cventry{2015}{Data Analysis and Statistical Inference}{Duke University}{Coursera}{}{Online course lectured by Mine Çetinkaya-Rundel.\newline{}}
\cventry{2014}{Algorithms: Design and Analysis, Part 2}{Stanford University}{Coursera}{}{Online course lectured by Tim Roughgarden.\newline{}}
\cventry{2014}{Algorithms: Design and Analysis, Part 1}{Stanford University}{Coursera}{}{Online course lectured by Tim Roughgarden.\newline{}}
\cventry{2013}{Principles of Reactive Programming}{École Polytechnique Fédérale de Lausanne}{Coursera}{}{Online course lectured by Martin Odersky, Erik Meijer and Roland Kuhn.\newline{}}
\cventry{2012}{Functional Programming Principles in Scala}{École Polytechnique Fédérale de Lausanne}{Coursera}{}{Online course lectured by Martin Odersky.\newline{}}
\cventry{2012}{Machine Learning}{Stanford University}{Coursera}{}{Online course lectured by Andrew Ng.\newline{}}

\section{Computer skills}
\cvline{Programming Languages}{C, C++, Java, Scala, R, Common Lisp, Prolog, Scheme, PHP, Ruby, JavaScript}
\cvline{Markup Languages}{XML, XHTML, LaTeX, CSS}
\cvline{Database Management Systems}{MySQL, PostgreSQL, Microsoft SQL Server}
\cvline{Revision Control Systems}{CVS, Subversion, Git}
\cvline{Frameworks and Technologies}{Android, OpenGL, Hibernate, Java Swing, Stripes, Ruby on Rails, Common Lisp Object System}

\section{Publications}
\cvitem{2013}{Nicos Angelopoulos, Vítor Santos Costa, João Azevedo, Jan Wielemaker, Rui Camacho and Lodewyk Wessels, \textit{Integrative functional statistics in logic programming}, Proceedings of the 15th International Symposium on Practical Aspects of Declarative Languages (PADL'13). Rome, Italy.}
\cvitem{2009}{João Azevedo, Miguel Oliveira, Pedro Pacheco and Luís Paulo Reis, \textit{A Cooperative CiberMouse@RTSS08 Team}, Proceedings of the 14th Portuguese Conference on Artificial Intelligence: Progress in Artificial Intelligence (EPIA'09). Aveiro, Portugal.}

\section{Academic Honors}
\cvlistitem{``Pr\'{e}mio Incentivo'' in 2005/2006, awarded by University of Porto to the best students finishing the first year of its courses.}
\cvlistitem{Merit scholarships in 2006/2007, 2007/2008 and 2008/2009, awarded by ``Faculdade de Engenharia da Universidade do Porto'' to the top students of all courses.}
\cvlistitem{``Pr\'{e}mio Companhia Portuguesa de Computadores, Inform\'{a}tica e Sistemas'' in 2009/2010, awarded by ``CPCIS - SGPS - Companhia Portuguesa de Computadores, Inform\'{a}tica e Sistemas'' to the student with the best grade average of the Integrated Masters in Informatics and Computing Engineering at Faculdade de Engenharia da Universidade do Porto.}
\cvlistitem{4th, 3rd and 2nd places, in 2007, 2008 and 2009, at TIUP (``Torneio Inter-Universit\'{a}rio de Programac\~{a}o''), as a member of the Divide\_N\_Conquer (2007 and 2008) and Theorem (2009) teams.}
\cvlistitem{3rd, 4th and 2nd places, in 2007, 2008 and 2009, at MIUP (``Maratona Inter-Universit\'{a}ria de Programac\~{a}o''), as a member of the Divide\_N\_Conquer (2007 and 2008) and Theorem (2009) teams.}
\cvlistitem{Two 2nd places, in 2007 and 2008, at CPUP (``Concurso de Programac~{a}o da Universidade do Porto''), as a member of the Divide\_N\_Conquer team.}
\cvlistitem{11th, 24th and 19th places, in 2007, 2008 and 2009, at SWERC (South Western Regional ACM Programming Contest), as a member of the Divide\_N\_Conquer (2007 and 2008) and Theorem (2009) teams.}
\cvlistitem{4th place, in 2008, at the CiberRato contest.}
\cvlistitem{5th place in Portuguese's Microsoft Imagine Cup 2010 Software Design Contest.}

% Publications from a BibTeX file without multibib\renewcommand*{\bibliographyitemlabel}{\@biblabel{\arabic{enumiv}}}% for BibTeX numerical labels
%\nocite{*}
%\bibliographystyle{plain}
%\bibliography{publications}       % 'publications' is the name of a BibTeX file

% Publications from a BibTeX file using the multibib package
%\section{Publications}
%\nocitebook{book1,book2}
%\bibliographystylebook{plain}
%\bibliographybook{publications}   % 'publications' is the name of a BibTeX file
%\nocitemisc{Azevedo2009}
%\bibliographystylemisc{plain}
%\bibliographymisc{publications}   % 'publications' is the name of a BibTeX file

\end{document}


%% end of file `template_en.tex'.
